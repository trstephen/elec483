\section{Data}\label{sec:data}
Working versions of encoder and decoder scripts are available at \url{https://github.com/trstephen/elec483}.
The scripts will embed and extract a text watermark from a provided grayscale or color image.

To evaluate the effects on image quality  with spatial and frequency alterations, we compared the method described in Section~\ref{sec:impl} to one that embedds the binary sequence without the BCH code directly in the image grayscale values.
This spatial embedding method produced a PSNR of 49.3924, significantly less than the PSNR of 58.7447 with our frequency embedding method.\improvement{why?}

We also attempted to determined the best ``layer'' to embed the watermark in a color image.
We used a couple color images, each with RGB and YCbCr encoded variants.
The results in Table~\ref{tbl:color-psnr} show that embedding a watermark in any RGB layer causes less of a decrease in PSNR than embedding it in any YCbCR layer.\improvement{why?}

\begin{table}[htpb]
  \centering
  \caption{Effects on PSNR after embedding a watermark in a single layer}
  \label{tbl:color-psnr}
  \begin{tabular}{@{}lcl@{}}
    \toprule
    \multicolumn{1}{c}{Source Image} & Layer & \multicolumn{1}{c}{PSNR}     \\
    \midrule
    \multirow{3}{*}{peppersYCbCr}
       & Y   & 58.91533 \\
       & Cb  & 58.12800 \\
       & Cr  & 58.95000 \\
    &  &  \\
    \multirow{3}{*}{peppersRGB}
       & R   & 63.60164 \\
       & G   & 63.74127 \\
       & B   & 63.70763 \\
    &  &  \\
    \multirow{3}{*}{fruitsYCbCr}
      & Y   & 58.88691 \\
      & Cb  & 57.95389 \\
      & Cr  & 59.07317 \\
    &  &  \\
    \multirow{3}{*}{fruitsRGB}
      & R   & 63.88637 \\
      & G   & 63.69708 \\
      & B   & 63.59464 \\
    \bottomrule
  \end{tabular}
\end{table}
