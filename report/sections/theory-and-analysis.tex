\section{Theory and analysis}\label{sec:t-and-a}
Unless specifically designed to be otherwise, the largest frequency component of an image is its DC component.
The DCT process typically yields a DC value over 1000. Altering this value by $\pm 1$ results in almost no change to image quality.
This is the core principle at the heart of many compression algorithms.\citeneeded{}

Relative image quality is determined by the Peak Signal-to-Noise Ratio (PSNR) value, representing the effect of noise from the $DCT~\rightarrow~IDCT~\rightarrow~DCT$ process and distortion from the watermark.

A Bose–Chaudhuri–Hocquenghem (BCH) code is a type of cyclic error correcting code.
It is defined by $k$, the length of the message, and $n$, the total code length.
For a given $n$, increasing $k$ decreases the number of bits that can be corrected, $t$.
