\section{Introduction}\label{sec:intro}
Digital watermarking is the process of embedding information in an image to establish ownership and prevent unauthorized distribution.
Traditional image watermarks that place translucent text or images over a source image must balance a degraded viewing experience with ease of removal.
Watermarks that are placed in the corner of an image rarely cover up important image features but can be removed trivially through cropping.
Prominent watermarks can discourage unauthorized image use (see Figure~\ref{fig:thumbs-up-wm}) but are impractical for use cases where a rights-holder wants to prevent dissemination of a high fidelity, minimally distorted image.

\begin{figure}[tbph]
  \centering
  \includegraphics[width=0.7\linewidth]{graphics/thumbs-up-wm}
  \caption{Stock image sites use overlay and footer watermarks to discourage unauthorized image use~\cite{stocklite:old-man}}
  \label{fig:thumbs-up-wm}
\end{figure}

Image metadata (e.g.\ EXIF, IIPC, PLUS, Dublin Core) exists in the digital file header and leaves the source image unaltered.
However, file metadata offers no protection against removal.\citeneeded{} An ideal digital watermark should be as inseparable from the source image as its namesake.

In this paper we demonstrate a procedure for watermarking digital files with almost no degradation of source image quality.
We take a plain language phrase and encode it in the image DCT blocks to make it an intrinsic part of the image that cannot be removed.
The watermark phrase is recovered by comparison with an original, unwatermarked image.
